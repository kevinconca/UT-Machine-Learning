Whether online or not when selling products it is useful to sort your products in categories. That way buyers have an easier time finding what they're looking for. It is tedious work to have to sort every item into a category by hand, especially if hundreds of new items are added at once. Unfortunately it is rather difficult to automate the sorting and placing of products in a real store but for online stores it should be possible to automatically sort a new product in one of the existing categories. 

The Otto Group is one of the world’s biggest e-commerce companies, with subsidiaries in more than 20 countries, including Crate \& Barrel (USA), Otto.de (Germany) and 3 Suisses (France). The company sells millions of products worldwide every day, with several thousand products being added to the product line. A consistent analysis of the performance of the products is crucial. However, due to our diverse global infrastructure, many identical products get classified differently. Therefore, the quality of our product analysis depends heavily on the ability to accurately cluster similar products. The better the classification, the more insights we can generate about our product range.

There are multiple learning algorithms which could help with this task, each with their own strengths and the trick will be to determine which algorithm will be most suitable for the classification of Otto products. So the question that needs to be asked is:
\begin{center}
\textbf{Which learning algorithm is most suited for the sorting of products into categories and with what settings will the performance be optimal?}
\end{center}
To answer this question, we also pose several sub-questions to help define the criteria of the learning algorithm.
\begin{center}
\begin{enumerate}
\item How accurate can the algorithm be? How many items will be wrongly classified?
\item How long will the trained system need to categorize a new item?
\end{enumerate}
\end{center}
Research will be done to find promising learning algorithms for this specific situation. The top ones will be implemented, trained and tested. For each of these the above sub-questions will be answered. Based on the answers it will be discussed which of the tested algorithms is best suited for the classifying of Otto products.